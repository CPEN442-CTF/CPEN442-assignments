\documentclass[manuscript, nonacm]{acmart}
\usepackage{fancyhdr}

%%%% Removes undesired artifacts of the ACM template
\makeatletter
\let\@authorsaddresses\@empty
\makeatother

%%%% Title
% TODO: Replace X with the correct assignment number
\title[CPEN442: Introdution to Computer Security - Fall 2020]{Assignment X Solution}
\subtitle{\today}

%%%% Authors
% TODO: For individual assignments, replace 12345 with the 5 left-most digits of your student number. For group assignments, you should use the group name you chose on Canvas, instead.
\author{Author: 12345}

%%% Start of the document
\begin{document}
\maketitle
\thispagestyle{fancy}

\section{Problem 1}
Follow instructions on the course website for what you have to provide for each problem. 

\section{Problem 2}
This is an example of how you should use citations: This template is based on the offical ACM proceedings template~\cite{ACMTemplate}. For more information on how to use the template, please refer to \url{https://www.acm.org/binaries/content/assets/publications/consolidated-tex-template/acmart.pdf}. Pay close attention to the Latex comments in this file that starts with "TODO."

\section{Problem 3}
Solution to problem 3

%%% References
% TODO: You should use the ACM format for your references. Put your BibTex definitions in the references.bib file. Make sure they are complete (e.g., page numbers, conference and journal titles, author names). Aim to use more academic references than non-academic ones.
\bibliographystyle{ACM-Reference-Format}
\bibliography{references}

%%% End of the document
\end{document}
